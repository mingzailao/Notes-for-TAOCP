% Created 2016-09-15 Thu 16:43
\documentclass[11pt]{article}
\usepackage[utf8]{inputenc}
\usepackage[T1]{fontenc}
\usepackage{fixltx2e}
\usepackage{graphicx}
\usepackage{longtable}
\usepackage{float}
\usepackage{wrapfig}
\usepackage{rotating}
\usepackage[normalem]{ulem}
\usepackage{amsmath}
\usepackage{textcomp}
\usepackage{marvosym}
\usepackage{wasysym}
\usepackage{amssymb}
\usepackage{hyperref}
\tolerance=1000
\author{apple}
\date{\today}
\title{notes}
\hypersetup{
  pdfkeywords={},
  pdfsubject={},
  pdfcreator={Emacs 24.5.1 (Org mode 8.2.10)}}
\begin{document}

\maketitle
\tableofcontents

\section{Basic Algorithm}
\label{sec-1}
\subsection{Algorithm}
\label{sec-1-1}
\subsubsection{Euclid Algorithm}
\label{sec-1-1-1}
Given 2 number m and n, request for their bcommon divisor
\begin{enumerate}
\item set r=m\%n
\item if r=0 Done return n
\item if r!=0 $m\leftarrow n$, $n\leftarrow r$ , return E1
\end{enumerate}
\begin{enumerate}
\item Code
\label{sec-1-1-1-1}

\begin{verbatim}
//Eculid.h
int Euclid(int m,int n){
  int r=m%n;
  while(r!=0){
    m=n;
    n=r;
    r=m%n;
  }
  return n;
}
// Euclid.cpp
#include<iostream>
#include "Euclid.h"
using namespace std;
int main()
{
    int m,n;
    cout<<"input 2 number"<<endl;
    cin>>m>>n;
    cout<<"run the function"<<endl;
    cout<<Euclid(m,n)<<endl;
    return 0;
}
\end{verbatim}
\end{enumerate}

\subsubsection{A Variant of Euclid Algorithm}
\label{sec-1-1-2}
\begin{enumerate}
\item if $m\ge n$, exchange m,n ($m\leftrightarrow n$)
\item set r=m\%n
\item if r=0 Done return n
\item if r!=0 $m\leftarrow n$, $n\leftarrow r$ , return E2
\end{enumerate}
\begin{enumerate}
\item Code
\label{sec-1-1-2-1}
\begin{verbatim}
//Functional file
int Variant_Euclid(int m,int n){
  if(m>n){
    int tmp=n;
    n=m;
    m=tmp;
  }
  int r=m%n;
  while(r!=0){
    m=n;
    n=r;
    r=m%n;
  }
  return n;
}
//Interface
#include<iostream>
#include "Variant_Euclid.h"

using std::cin;
using std::cout;
using std::endl;
int main()
{
    cout<<"Please input 2 number"<<endl;
    int m,n;
    cin>>m>>n;
    cout<<Variant_Euclid(m,n)<<endl;
    return 0;
}
\end{verbatim}
\end{enumerate}

\subsubsection{Exercise}
\label{sec-1-1-3}
\begin{itemize}
\item 1
\end{itemize}
$t \leftarrow a$, $a \leftarrow b$, $b \leftarrow c$, $c \leftarrow d$, $a \leftarrow t$.
\begin{itemize}
\item 2. prove that after step 2, $m>n$
\end{itemize}

solution

In step one, $r=m\%n$, if $r\neq 0$ (r<n),
then $m\leftarrow n$,
$n\leftarrow r$,
so $m>n$
\begin{itemize}
\item 4.
\end{itemize}

Solution

57
\begin{itemize}
\item 5.
\end{itemize}

Finiteness: the reader can work on exercices and check answers indefinitely

Output: the procedure have no output

Effectiveness: relaxing and sleeping are not effective operations
\begin{itemize}
\item 6.
\end{itemize}

N0 = 1; N1 = 2; N2 = 3; N3 = 4; N4 = 3;

T5 = (1+2+3+4+3)/5 = 13/5 = 2.6

If k < 5, the first iteration just permutes the values of m and n.

And if k >= 5, the first iteration is a division of k by 5.

So the number of steps is the same for each class modulo 5 of integers.
\begin{itemize}
\item 7.
\end{itemize}

$U_n=T_{n+1}$

\subsection{Mathematical Foundation}
\label{sec-1-2}
\subsubsection{Mathematical Induction(MI)}
\label{sec-1-2-1}
\begin{enumerate}
\item $P(1)$ is true.
\item if $P(2),P(3),\cdots,P(n)$ is true then $P(n+1)$ is True.
\end{enumerate}
\subsubsection{Example}
\label{sec-1-2-2}
\begin{equation}
\label{eq:1}
1+3+5+7+9=5^2
\end{equation}
then 
\begin{equation}
\label{eq:2}
1+3+\cdots+(2n-1)=n^2
\end{equation}
Call this equation $P(n)$

\rule{\linewidth}{0.5pt}
\begin{enumerate}
\item $\because$ 1=1$^{\text{2}}$ so P(1) is True
\item if P(1),P(2),P(3),$\cdots{}$,P(n) is True, especially P(n) is True then Eq. (\ref{eq:2}), add (2n+1) on both sides of the equation:
\end{enumerate}
\begin{equation*}
1+3+\cdots+(2n-1)+(2n+1)=n^2+2n+1=(n+1)^2
\end{equation*}
we get P(n+1) is True.

\rule{\linewidth}{0.5pt}
% Emacs 24.5.1 (Org mode 8.2.10)
\end{document}